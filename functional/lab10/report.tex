\documentclass[a4paper,oneside,12pt]{extreport}

\include{preamble}

\begin{document}

\begin{titlepage}
	{\large % 14pt instead of 12pt
	\onehalfspacing
	\centering

	\begin{wrapfigure}[7]{l}{0.14\linewidth}
		\vspace{3mm}
		\hspace{-10mm}
		\includegraphics[width=0.93\linewidth]{inc/img/bmstu-logo}
	\end{wrapfigure}
	{\singlespacing \footnotesize \bfseries Министерство науки и высшего образования Российской Федерации\\Федеральное государственное бюджетное образовательное учреждение\\высшего образования\\<<Московский государственный технический университет\\имени Н.~Э.~Баумана\\ (национальный исследовательский университет)>>\\(МГТУ им. Н.~Э.~Баумана)\\}

	\vspace{-2.2mm}
	\vhrulefill{0.9mm}\\
	\vspace{-7.5mm}
	\vhrulefill{0.2mm}\\
	\vspace{2mm}

	{\doublespacing \small \raggedright ФАКУЛЬТЕТ \hspace{25mm} «Информатика и системы управления»\\
	КАФЕДРА \hspace{5mm} «Программное обеспечение ЭВМ и информационные технологии»\\}

	\vspace{30mm}

	\textbf{ОТЧЁТ}\\
	По лабораторной работе №3\\
	По курсу: «Функциональное и логическое программирование»\\
	%Тема: «Тема работы»\\

	\vspace{60mm}

	\hspace{70mm} Студент:       \hfill Керимов~А.~Ш.\\
	\hspace{70mm} Группа:        \hfill ИУ7-64Б\\
	\hspace{70mm} Преподаватели: \hfill Толпинская~Н.~Б.,\\
	                             \hfill Строганов~Ю.~В.\\

	\vfill

	Москва\\
	\the\year\\}
\end{titlepage}


\section*{Практическая часть}

\begin{task}
	Пусть \code{list-of-list} список, состоящий из списков.
	Написать функцию, которая вычисляет сумму длин всех элементов \code{list-of-list}, т.~е. например для аргумента \code{((1 2) (3 4)) -> 4}.

	\begin{lstlisting}[language=Lisp, gobble=16]
		(defun sum-of-lengths-process (list-of-list acc)
		  (if list-of-list
		      (sum-of-lengths-process
		        (cdr list-of-list)
		        (+ acc (length (car list-of-list))))
		      acc))

		(defun sum-of-lengths (list-of-list)
		  (sum-of-lengths-process list-of-list 0))
	\end{lstlisting}

	Функция \code{sum-of-lengths-process} принимает список списков \code{list-of-list} и накопленную сумму \code{acc}.
	Если дошли до конца списка, возвращает накопленную сумму.
	Иначе, вычисляется рекурсивно для хвоста и накопленной суммы, к которой прибавлена длина списка в голове \code{list-of-list}.

	Примеры работы:
	\begin{table}[H]
		\begin{center}
			\begin{tabular}{|l|l|}
				\hline
				\textbf{list-of-list} & \textbf{(sum-of-lengths list-of-list)} \\ \hline
				\code{((1 2) (3 4))} & \code{4} \\ \hline
				\code{((1 2 (3 3 3)) (4 5 (6 6 6)))} & \code{6} \\ \hline
			\end{tabular}
		\end{center}
	\end{table}
\end{task}

\begin{task}
	Написать рекурсивную версию (с именем \code{reg-add}) вычисления суммы чисел заданного списка.
	Например: \code{(reg-add (2 4 6)) -> 12}.

	\begin{lstlisting}[language=Lisp, gobble=16]
		(defun reg-add-process (lst acc)
		  (if lst
		      (reg-add-process
		        (cdr lst)
		        (+ acc (car lst)))
		      acc))

		(defun reg-add (lst)
		  (reg-add-process lst 0))
	\end{lstlisting}

	Функция \code{reg-add-process} принимает список чисел \code{lst} и накопленную сумму \code{acc}.
	Если дошли до конца списка, возвращает накопленную сумму.
	Иначе, вычисляется рекурсивно для хвоста и накопленной суммы, к которой прибавили число в голове списка.

	Примеры работы:
	\begin{table}[H]
		\begin{center}
			\begin{tabular}{|l|l|}
				\hline
				\textbf{lst} & \textbf{(reg-add lst)} \\ \hline
				\code{(1 2 3 4)} & \code{10} \\ \hline
				\code{(1 2 3 4 5)} & \code{15} \\ \hline
				\code{()} & \code{0} \\ \hline
			\end{tabular}
		\end{center}
	\end{table}
\end{task}

\begin{task}
	Написать рекурсивную версию с именем \code{recnth} функции \code{nth}.

	\begin{lstlisting}[language=Lisp, gobble=16]
		(defun recnth (n lst)
		  (cond ((null lst) nil)
		        ((= n 0) (car lst))
		        (t (recnth (- n 1) (cdr lst)))))
	\end{lstlisting}

	Функция \code{recnth} принимает индекс \code{n} и список \code{lst}.
	Если дошли до конца списка, возвращает \code{nil}.
	Иначе, если $n = 0$, возвращает голову списка.
	Иначе, вычисляется рекурсивно для $n - 1$ и хвоста списка.

	Примеры работы:
	\begin{table}[H]
		\begin{center}
			\begin{tabular}{|l|l|l|}
				\hline
				\textbf{n} & \textbf{lst} & \textbf{(recnth n lst)} \\ \hline
				\code{2} & \code{(1 2 3 4)} & \code{3} \\ \hline
				\code{4} & \code{(1 2 3 4)} & \code{nil} \\ \hline
				\code{4} & \code{()} & \code{nil} \\ \hline
			\end{tabular}
		\end{center}
	\end{table}
\end{task}

\begin{task}
	Написать рекурсивную функцию \code{alloddr}, которая возвращает \code{t} когда все элементы списка нечетные.

	\begin{lstlisting}[language=Lisp, gobble=16]
		(defun alloddr-process (lst acc)
		  (if (or (null acc)
		          (null lst))
		      acc
		      (alloddr-process
		        (cdr lst)
		        (oddp (car lst)))))

		(defun alloddr (lst)
		  (alloddr-process lst t))
	\end{lstlisting}

	Функция \code{alloddr-process} принимает список чисел \code{lst} и результат \code{acc}.
	Если дошли до конца списка или результат \code{nil}, возвращает результат.
	Иначе, вычисляется рекурсивно для хвоста и результата, который равен \code{t}, если голова списка является нечётным числом, и \code{nil} иначе.

	Примеры работы:
	\begin{table}[H]
		\begin{center}
			\begin{tabular}{|l|l|}
				\hline
				\textbf{lst} & \textbf{(alloddr lst)} \\ \hline
				\code{(1 2 3 4 5)} & \code{nil} \\ \hline
				\code{(1 3 5)} & \code{t} \\ \hline
				\code{(1 3 5 4)} & \code{nil} \\ \hline
				\code{()} & \code{nil} \\ \hline
			\end{tabular}
		\end{center}
	\end{table}
\end{task}

\begin{task}
	Написать рекурсивную функцию, относящуюся к хвостовой рекурсии с одним тестом завершения, которая возвращает последний элемент списка — аргументы.

	\begin{lstlisting}[language=Lisp, gobble=16]
		(defun rec-last (lst)
		  (if (cdr lst)
		      (rec-last (cdr lst))
		      (car lst)))
	\end{lstlisting}

	Функция \code{rec-last} принимает список \code{lst}.
	Если хвост списка не пустой, функция вычисляется рекурсивно для него (хвоста списка).
	Иначе, возвращает голову списка (последний элемент).

	Примеры работы:
	\begin{table}[H]
		\begin{center}
			\begin{tabular}{|l|l|}
				\hline
				\textbf{lst} & \textbf{(rec-last lst)} \\ \hline
				\code{(1 2 3 4)} & \code{4} \\ \hline
				\code{(1 2 3 4 5)} & \code{5} \\ \hline
				\code{()} & \code{nil} \\ \hline
			\end{tabular}
		\end{center}
	\end{table}
\end{task}

\begin{task}
	Написать рекурсивную функцию, относящуюся к дополняемой рекурсии с одним тестом завершения, которая вычисляет сумму всех чисел от 0 до $n$-ого аргумента функции.
	Вариант: 1) от $n$-аргумента функции до последнего >= 0.

	\begin{lstlisting}[language=Lisp, gobble=16]
		(defun sum-to-n (lst n)
		  (if (and lst (> n 0))
		      (+ (car lst)
		         (sum-to-n (cdr lst) (- n 1)))
		      0))

		(defun sum-from-n (lst n)
		  (if lst
		      (+
		        (if (<= n 0) (car lst) 0)
		        (sum-from-n (cdr lst) (- n 1)))
		      0))
	\end{lstlisting}

	Функция \code{sum-to-n} принимает список \code{lst}.
	Если список не пустой и $n > 0$, возвращает сумму головы списка с результатом рекурсивного вычисления функции для хвоста списка и $n - 1$.
	Иначе, возвращает 0.

	Функция \code{sum-from-n} принимает список \code{lst}.
	Если список пустой, возвращает 0.
	Иначе, возвращает сумму, где первое слагаемое является головой списка в случае, если $n < 0$, и нулём иначе, а второе слагаемое — результатом рекурсивного вычисления функции для хвоста списка и $n - 1$.

	Примеры работы:
	\begin{table}[H]
		\begin{center}
			\begin{tabular}{|l|l|l|l|}
				\hline
				\textbf{lst} & \textbf{n} & \textbf{(sum-to-n lst n)} & \textbf{(sum-from-n lst n)} \\ \hline
				\code{(1 2 3 4 5)} & \code{2} &\code{3}  & \code{12} \\ \hline
				\code{(1 2 3 4 5)} & \code{5} &\code{15} & \code{0}  \\ \hline
				\code{(1 2 3 4 5)} & \code{0} &\code{0}  & \code{15} \\ \hline
			\end{tabular}
		\end{center}
	\end{table}
\end{task}

\begin{task}
	Написать рекурсивную функцию, которая возвращает последнее нечетное число из числового списка, возможно создавая некоторые вспомогательные функции.

	\begin{lstlisting}[language=Lisp, gobble=16]
		(defun last-odd-process (lst acc)
		  (if lst
		      (last-odd-process
		        (cdr lst)
		        (if (oddp (car lst))
		            (car lst)
		            acc))
		      acc))

		(defun last-odd (lst)
		  (last-odd-process lst nil))
	\end{lstlisting}

	Функция \code{last-odd-process} принимает список \code{lst} и результат \code{acc}.
	Если дошли до конца списка, возвращает результат.
	Иначе, вычисляется рекурсивно для хвоста и результата, который равен голове списка, если она (голова списка) является нечётным числом, или прежнему результату иначе.

	Примеры работы:
	\begin{table}[H]
		\begin{center}
			\begin{tabular}{|l|l|}
				\hline
				\textbf{lst} & \textbf{(last-odd lst)} \\ \hline
				\code{(1 2 3 4)} & \code{3} \\ \hline
				\code{(1 2 3 4 5)} & \code{5} \\ \hline
				\code{(2 4 6)} & \code{nil} \\ \hline
				\code{()} & \code{nil} \\ \hline
			\end{tabular}
		\end{center}
	\end{table}
\end{task}

\begin{task}
	Используя \code{cons}-дополняемую рекурсию с одним тестом завершения, написать функцию которая получает как аргумент список чисел, а возвращает список квадратов этих чисел в том же порядке.

	\begin{lstlisting}[language=Lisp, gobble=16]
		(defun sqr (number)
		  (* number number))

		(defun sqr-lst (lst)
		  (if lst
		      (cons (sqr (car lst))
		            (sqr-lst (cdr lst)))
		      nil))
	\end{lstlisting}

	Функция \code{sqr-lst} принимает список \code{lst}.
	Если список не пустой и возвращает список, голова которого является квадратом исходной головы списка, а хвост — результатом рекурсивного вычисления функции для хвоста исходного списка.
	Иначе, возвращает \code{nil}.

	Примеры работы:
	\begin{table}[H]
		\begin{center}
			\begin{tabular}{|l|l|}
				\hline
				\textbf{lst} & \textbf{(last-odd lst)} \\ \hline
				\code{(1 2 3 4)} & \code{(1 4 9 16)} \\ \hline
				\code{()} & \code{()} \\ \hline
			\end{tabular}
		\end{center}
	\end{table}
\end{task}

\begin{task}
	Написать функцию с именем \code{select-odd}, которая из заданного списка выбирает все нечетные числа.
	(Вариант 1: \code{select-even}, вариант 2: вычисляет сумму всех нечетных чисел (\code{sum-all-odd}) или сумму всех четных чисел (\code{sum-all-even}) из заданного списка)

	\begin{lstlisting}[language=Lisp, gobble=16]
		(defun select-odd (lst)
		  (if lst
		      (if (oddp (car lst))
		          (cons (car lst)
		                (select-odd (cdr lst)))
		          (select-odd (cdr lst)))
		      nil))

		(defun select-even (lst)
		  (if lst
		      (if (evenp (car lst))
		          (cons (car lst)
		                (select-even (cdr lst)))
		          (select-even (cdr lst)))
		      nil))

		(defun sum-all-odd-process (lst acc)
		  (if lst
		      (sum-all-odd-process
		        (cdr lst)
		        (if (oddp (car lst))
		            (+ acc (car lst))
		            acc))
		      acc))

		(defun sum-all-odd (lst)
		  (sum-all-odd-process lst 0))

		(defun sum-all-even-process (lst acc)
		  (if lst
		      (sum-all-even-process
		        (cdr lst)
		        (if (evenp (car lst))
		            (+ acc (car lst))
		            acc))
		      acc))

		(defun sum-all-even (lst)
		  (sum-all-even-process lst 0))
	\end{lstlisting}

	Примеры работы:
	\begin{table}[H]
		\begin{center}
			\begin{tabular}{|l|l|l|l|l|}
				\hline
				\textbf{lst} & \textbf{(select-odd lst)} & \textbf{(select-even lst)} & \textbf{(sum-all-odd lst)} & \textbf{(sum-all-even lst)} \\ \hline
				\code{(1 2 3 4 5)} & \code{(1 3 5)} & \code{(2 4)}  & \code{9} & \code{6} \\ \hline
				\code{(1 3 5)} & \code{(1 3 5)} & \code{()}  & \code{9} & \code{0} \\ \hline
				\code{()} & \code{()} &\code{()}  & \code{0} & \code{0} \\ \hline
			\end{tabular}
		\end{center}
	\end{table}
\end{task}

\section*{Теоретическая часть}

\subsection*{Способы организации повторных вычислений в Lisp}

Для организации повторных вычислений в Lisp могут быть использованы функционалы и рекурсия.

\subsection*{Что такое рекурсия? Классификация рекурсивных функций в Lisp}

Рекурсия — это ссылка на определяемый объект во время его определения.

В Lisp существует классификация рекурсивных функций:
\begin{itemize}
	\item простая рекурсия — один рекурсивный вызов в теле;
	\item рекурсия первого порядка — рекурсивный вызов встречается несколько раз;
	\item взаимная рекурсия — используется несколько функций, рекурсивно вызывающих друг друга.
\end{itemize}

\subsection*{Различные способы организации рекурсивных функций и порядок их реализации}

При организации рекурсии можно использовать как функции с именем, так и локально определенные с помощью лямбда-выражений функции.
Кроме этого, при организации рекурсии можно использовать функционалы или использовать рекурсивную функцию внутри функционала.
При изучении рекурсии рекомендуется организовывать и отлаживать реализацию отдельных подзадач исходной задачи, обращая внимание на эффективность реализации и качество работы, а потом, при необходимости, встраивать эти функции в более крупные, возможно в виде лямбда-выражений.

\subsection*{Способы повышения эффективности реализации рекурсии}

\begin{itemize}
	\item {\bfseries Хвостовая рекурсия}.
	В целях повышения эффективности рекурсивных функций рекомендуется формировать результат не на выходе из рекурсии, а на входе в рекурсию, все действия выполняя до ухода на следующий шаг рекурсии.

	Преобразование не хвостовой рекурсии в хвостовую, возможно путем использования дополнительных параметров.
	В этом случае необходимо использовать функцию-оболочку для запуска рекурсивной функции с начальными значениями дополнительных параметров.

	\item {\bfseries Дополняемая рекурсия}.
	При обращении к рекурсивной функции используется дополнительная функция не в аргументе вызова, а вне его.

	\item {\bfseries Выделяют группу функций множественной рекурсии}.
	На одной ветке происходит сразу несколько рекурсивных вызовов.
	Количество условий выхода также может зависеть от задачи.
\end{itemize}

\end{document}
tell me you love me youll get i love you more