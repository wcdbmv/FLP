\documentclass[a4paper,oneside,12pt]{extreport}

\include{preamble}

\begin{document}

\begin{titlepage}
	{\large % 14pt instead of 12pt
	\onehalfspacing
	\centering

	\begin{wrapfigure}[7]{l}{0.14\linewidth}
		\vspace{3mm}
		\hspace{-10mm}
		\includegraphics[width=0.93\linewidth]{inc/img/bmstu-logo}
	\end{wrapfigure}
	{\singlespacing \footnotesize \bfseries Министерство науки и высшего образования Российской Федерации\\Федеральное государственное бюджетное образовательное учреждение\\высшего образования\\<<Московский государственный технический университет\\имени Н.~Э.~Баумана\\ (национальный исследовательский университет)>>\\(МГТУ им. Н.~Э.~Баумана)\\}

	\vspace{-2.2mm}
	\vhrulefill{0.9mm}\\
	\vspace{-7.5mm}
	\vhrulefill{0.2mm}\\
	\vspace{2mm}

	{\doublespacing \small \raggedright ФАКУЛЬТЕТ \hspace{25mm} «Информатика и системы управления»\\
	КАФЕДРА \hspace{5mm} «Программное обеспечение ЭВМ и информационные технологии»\\}

	\vspace{30mm}

	\textbf{ОТЧЁТ}\\
	По лабораторной работе №3\\
	По курсу: «Функциональное и логическое программирование»\\
	%Тема: «Тема работы»\\

	\vspace{60mm}

	\hspace{70mm} Студент:       \hfill Керимов~А.~Ш.\\
	\hspace{70mm} Группа:        \hfill ИУ7-64Б\\
	\hspace{70mm} Преподаватели: \hfill Толпинская~Н.~Б.,\\
	                             \hfill Строганов~Ю.~В.\\

	\vfill

	Москва\\
	\the\year\\}
\end{titlepage}


\begin{task}
	Пусть \code{list-of-list} список, состоящий из списков.
	Написать функцию, которая вычисляет сумму длин всех элементов \code{list-of-list}, т.~е. например для аргумента \code{((1 2) (3 4)) -> 4}.

	\begin{lstlisting}[language=Lisp, gobble=16]
		(defun sum-of-lengths (list-of-list)
		  (defun sum-of-lengths-process (list-of-list acc)
		    (if list-of-list
		        (sum-of-lengths-process
		          (cdr list-of-list)
		          (+ acc
		             (if (atom (car list-of-list))
		                 1
		                 (length (car list-of-list)))))
		        acc))
		  (sum-of-lengths-process list-of-list 0))
	\end{lstlisting}
\end{task}

\begin{task}
	Написать рекурсивную версию (с именем \code{reg-add}) вычисления суммы чисел заданного списка.
	Например: \code{(reg-add (2 4 6)) -> 12}.

	\begin{lstlisting}[language=Lisp, gobble=16]
		(defun reg-add (lst)
		  (defun reg-add-process (lst acc)
		    (if lst
		        (reg-add-process
		          (cdr lst)
		          (+ acc (car lst)))
		        acc))
		  (reg-add-process lst 0))
	\end{lstlisting}
\end{task}

\begin{task}
	Написать рекурсивную версию с именем \code{recnth} функции \code{nth}.

	\begin{lstlisting}[language=Lisp, gobble=16]
		(defun recnth (n lst)
		  (cond ((null lst) nil)
		        ((= n 0) (car lst))
		        (t (recnth (- n 1) (cdr lst)))))
	\end{lstlisting}
\end{task}

\begin{task}
	Написать рекурсивную функцию \code{alloddr}, которая возвращает \code{t} когда все элементы списка нечетные.

	\begin{lstlisting}[language=Lisp, gobble=16]
		(defun alloddr (lst)
		  (defun alloddr-process (lst acc)
		    (if (or (null acc)
		            (null lst))
		        acc
		        (alloddr-process
		          (cdr lst)
		          (oddp (car lst)))))
		  (alloddr-process lst t))
	\end{lstlisting}
\end{task}

\begin{task}
	Написать рекурсивную функцию, относящуюся к хвостовой рекурсии с одним тестом завершения, которая возвращает последний элемент списка — аргументы.

	\begin{lstlisting}[language=Lisp, gobble=16]
		(defun rec-last (lst)
		  (if (cdr lst)
		      (rec-last (cdr lst))
		      (car lst)))
	\end{lstlisting}
\end{task}

\begin{task}
	Написать рекурсивную функцию, относящуюся к дополняемой рекурсии с одним тестом завершения, которая вычисляет сумму всех чисел от 0 до $n$-ого аргумента функции.
	Вариант:
	\begin{enumerate}
		\item от $n$-аргумента функции до последнего >= 0,
		\item от $n$-аргумента функции до $t$-аргумента с шагом $d$.
	\end{enumerate}

	\begin{lstlisting}[language=Lisp, gobble=16]
		(defun sum-to-n (lst n)
		  (if (and lst (> n 0))
		      (+ (car lst)
		         (sum-to-n (cdr lst) (- n 1)))
		      0))

		(defun sum-from-n (lst n)
		  (if lst
		      (+
		        (if (<= n 0) (car lst) 0)
		        (sum-from-n (cdr lst) (- n 1)))
		      0))
	\end{lstlisting}
\end{task}

\begin{task}
	Написать рекурсивную функцию, которая возвращает последнее нечетное число из числового списка, возможно создавая некоторые вспомогательные функции.

	\begin{lstlisting}[language=Lisp, gobble=16]
		(defun last-odd (lst)
		  (defun last-odd-process (lst acc)
		    (if lst
		        (last-odd-process
		          (cdr lst)
		          (if (oddp (car lst))
		              (car lst)
		              acc))
		        acc))
		  (last-odd-process lst nil))
	\end{lstlisting}
\end{task}

\begin{task}
	Используя \code{cons}-дополняемую рекурсию с одним тестом завершения, написать функцию которая получает как аргумент список чисел, а возвращает список квадратов этих чисел в том же порядке.

	\begin{lstlisting}[language=Lisp, gobble=16]
		(defun sqr-lst (lst)
		  (if lst
		      (cons (sqr (car lst))
		            (sqr-lst (cdr lst)))
		      nil))
	\end{lstlisting}
\end{task}

\begin{task}
	Написать функцию с именем \code{select-odd}, которая из заданного списка выбирает все нечетные числа.
	(Вариант 1: \code{select-even}, вариант 2: вычисляет сумму всех нечетных чисел (\code{sum-all-odd}) или сумму всех четных чисел (\code{sum-all-even}) из заданного списка)

	\begin{lstlisting}[language=Lisp, gobble=16]
		(defun select-odd (lst)
		  (if lst
		      (if (oddp (car lst))
		          (cons (car lst)
		                (select-odd (cdr lst)))
		          (select-odd (cdr lst)))
		      nil))

		(defun select-even (lst)
		  (if lst
		      (if (evenp (car lst))
		          (cons (car lst)
		                (select-even (cdr lst)))
		          (select-even (cdr lst)))
		      nil))

		(defun sum-all-odd (lst)
		  (defun sum-all-odd-process (lst acc)
		    (if lst
		        (sum-all-odd-process
		          (cdr lst)
		          (if (oddp (car lst))
		              (+ acc (car lst))
		              acc))
		        acc))
		  (sum-all-odd-process lst 0))

		(defun sum-all-even (lst)
		  (defun sum-all-even-process (lst acc)
		    (if lst
		        (sum-all-even-process
		          (cdr lst)
		          (if (evenp (car lst))
		              (+ acc (car lst))
		              acc))
		        acc))
		  (sum-all-even-process lst 0))
	\end{lstlisting}
\end{task}

\end{document}
