\documentclass[a4paper,oneside,12pt]{extreport}

\include{preamble}

\begin{document}

\begin{titlepage}
	{\large % 14pt instead of 12pt
	\onehalfspacing
	\centering

	\begin{wrapfigure}[7]{l}{0.14\linewidth}
		\vspace{3mm}
		\hspace{-10mm}
		\includegraphics[width=0.93\linewidth]{inc/img/bmstu-logo}
	\end{wrapfigure}
	{\singlespacing \footnotesize \bfseries Министерство науки и высшего образования Российской Федерации\\Федеральное государственное бюджетное образовательное учреждение\\высшего образования\\<<Московский государственный технический университет\\имени Н.~Э.~Баумана\\ (национальный исследовательский университет)>>\\(МГТУ им. Н.~Э.~Баумана)\\}

	\vspace{-2.2mm}
	\vhrulefill{0.9mm}\\
	\vspace{-7.5mm}
	\vhrulefill{0.2mm}\\
	\vspace{2mm}

	{\doublespacing \small \raggedright ФАКУЛЬТЕТ \hspace{25mm} «Информатика и системы управления»\\
	КАФЕДРА \hspace{5mm} «Программное обеспечение ЭВМ и информационные технологии»\\}

	\vspace{30mm}

	\textbf{ОТЧЁТ}\\
	По лабораторной работе №3\\
	По курсу: «Функциональное и логическое программирование»\\
	%Тема: «Тема работы»\\

	\vspace{60mm}

	\hspace{70mm} Студент:       \hfill Керимов~А.~Ш.\\
	\hspace{70mm} Группа:        \hfill ИУ7-64Б\\
	\hspace{70mm} Преподаватели: \hfill Толпинская~Н.~Б.,\\
	                             \hfill Строганов~Ю.~В.\\

	\vfill

	Москва\\
	\the\year\\}
\end{titlepage}


\begin{task}
	Написать функцию, которая по своему списку-аргументу \code{lst} определяет, является ли он палиндромом (то есть равны ли \code{lst} и \code{(reverse lst)}).

	\begin{lstlisting}[language=Lisp, gobble=16]
		(defun is_palindrom (lst)
		  (and (equal (null lst) nil)
		       (equal (reverse lst) lst)))
	\end{lstlisting}
\end{task}

\begin{task}
	Написать предикат \code{set-equal}, который возвращает \code{t}, если два его множества-аргумента содержат одни и те же элементы, порядок которых не имеет значения.
	\begin{lstlisting}[language=Lisp, gobble=16]
		(defun set-equal1 (a b)
		  (and (null (set-difference a b))
		       (null (set-difference b a))))

		(defun set-equal2 (a b)
		  (not (set-exclusive-or a b)))
	\end{lstlisting}
\end{task}

\begin{task}
	Напишите функцию \code{swap-first-last}, которая переставляет в списке-аргументе первый и последний элементы.

	\begin{lstlisting}[language=Lisp, gobble=16]
		(defun swap-first-last (x)
		  (cond ((null (cdr x)) x)
		        (t (append (last x) (cdr (butlast x)) (list (car x))))))
	\end{lstlisting}
\end{task}

\begin{task}
	Напишите функцию \code{swap-two-elements}, которая переставляет в списке- аргументе два указанных своими порядковыми номерами элемента в этом списке.

	\begin{lstlisting}[language=Lisp, gobble=16]
		(defun swap-two-elements1 (x i1 i2)
		  (cond ((null (rotatef (nth i1 x) (nth i2 x))) x)))

		(defun swap-two-elements2 (x i1 i2)
		  (let ((i (min i1 i2)) (j (max i1 i2)))
		       (append
		         (subseq x 0 i)
		         (list (nth j x))
		         (subseq x (+ i 1) j)
		         (list (nth i x))
		         (subseq x (+ j 1)))))

		(defun swap-two-elements3 (x i1 i2)
		  (let ((i (min i1 i2)) (j (+ (max i1 i2) 1)))
		    (append
		      (subseq x 0 i)
		      (swap-first-last (subseq x i j))
		      (subseq x j))))
	\end{lstlisting}
\end{task}

\begin{task}
	Напишите две функции, \code{swap-to-left} и \code{swap-to-right}, которые производят круговую перестановку в списке-аргументе влево и вправо, соответственно.

	\begin{lstlisting}[language=Lisp, gobble=16]
		(defun swap-to-left (lst &optional (k 1))
		  (let ((k (rem k (length lst))))
		       (append (subseq lst k) (subseq lst 0 k))))

		(defun swap-to-right (lst &optional (k 1))
		  (setf len (length lst))
		  (let ((k (- len (rem k len))))
		       (append (subseq lst k) (subseq lst 0 k))))
	\end{lstlisting}
\end{task}

\begin{task}
	Напишите функцию, которая умножает на заданное число-аргумент все числа из заданного списка-аргумента, когда
	a) все элементы списка — числа,
	б) элементы списка — любые объекты.

	\begin{lstlisting}[language=Lisp, gobble=16]
		(defun mul_n (lst n)
		  (mapcar (lambda (x) (* n x)) lst))

		(defun mul_all (lst n)
		  (mapcar
		    (lambda (x)
		      (cond
		        ((numberp x) (* n x))
		        ((listp x) (mul_all x n))
		        (t x)))
		    lst))
	\end{lstlisting}
\end{task}

\begin{task}
	Напишите функцию, \code{select-between}, которая из списка-аргумента,
	содержащего только числа, выбирает только те, которые расположены между двумя указанными границами-аргументами и возвращает их в виде списка (упорядоченного по возрастанию списка чисел).

	\begin{lstlisting}[language=Lisp, gobble=16]
		(defun select-between (lst a b)
		  (let ((a (min a b)) (b (max a b)))
		       (sort
		         (remove-if-not (lambda (x)
		           (and (<= a x) (<= x b))) lst)
		         #'<)))
	\end{lstlisting}
\end{task}

\end{document}
