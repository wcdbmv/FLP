\documentclass[a4paper,oneside,12pt]{extreport}

\include{preamble}

\begin{document}

\begin{titlepage}
	{\large % 14pt instead of 12pt
	\onehalfspacing
	\centering

	\begin{wrapfigure}[7]{l}{0.14\linewidth}
		\vspace{3mm}
		\hspace{-10mm}
		\includegraphics[width=0.93\linewidth]{inc/img/bmstu-logo}
	\end{wrapfigure}
	{\singlespacing \footnotesize \bfseries Министерство науки и высшего образования Российской Федерации\\Федеральное государственное бюджетное образовательное учреждение\\высшего образования\\<<Московский государственный технический университет\\имени Н.~Э.~Баумана\\ (национальный исследовательский университет)>>\\(МГТУ им. Н.~Э.~Баумана)\\}

	\vspace{-2.2mm}
	\vhrulefill{0.9mm}\\
	\vspace{-7.5mm}
	\vhrulefill{0.2mm}\\
	\vspace{2mm}

	{\doublespacing \small \raggedright ФАКУЛЬТЕТ \hspace{25mm} «Информатика и системы управления»\\
	КАФЕДРА \hspace{5mm} «Программное обеспечение ЭВМ и информационные технологии»\\}

	\vspace{30mm}

	\textbf{ОТЧЁТ}\\
	По лабораторной работе №3\\
	По курсу: «Функциональное и логическое программирование»\\
	%Тема: «Тема работы»\\

	\vspace{60mm}

	\hspace{70mm} Студент:       \hfill Керимов~А.~Ш.\\
	\hspace{70mm} Группа:        \hfill ИУ7-64Б\\
	\hspace{70mm} Преподаватели: \hfill Толпинская~Н.~Б.,\\
	                             \hfill Строганов~Ю.~В.\\

	\vfill

	Москва\\
	\the\year\\}
\end{titlepage}


\section*{Практическая часть}

\begin{task}
	Написать предикат \code{set-equal}, который возвращает \code{t}, если два его множество-аргумента содержат одни и те же элементы, порядок которых не имеет значения.

	\begin{lstlisting}[language=Lisp, gobble=16]
		(defun set-equal (a b)
		  (null (set-exclusive-or a b)))
	\end{lstlisting}

	\begin{lstlisting}[language=Lisp, gobble=16]
		(defun fset-equal (a b)
		  (if (= (list-length a) (list-length b))
		      (reduce
		        #'(lambda (acc cur)
		            (if acc (member cur b) nil))
		        a
		        :initial-value t)
		      nil))
	\end{lstlisting}

	\begin{lstlisting}[language=Lisp, gobble=16]
		(defun rset-equal (a b)
		  (defun rset-equal-process (a)
		    (cond ((null a) t)
		          ((member (car a) b) (rset-equal-process (cdr a)))
		          (t nil)))
		  (if (= (list-length a) (list-length b))
		      (rset-equal-process a)
		      nil))
	\end{lstlisting}

	Примеры работы:
	\begin{table}[H]
		\begin{center}
			\begin{tabular}{|l|l|l|l|}
				\hline
				\textbf{a} & \textbf{b} & \textbf{(fset-equal a b)} & \textbf{(rset-equal a b)} \\ \hline
				\code{(1 2 3)} & \code{(3 2 1)} & \code{(3 2 1)}  & \code{t}   \\ \hline
				\code{(1 2 3)} & \code{(3 2)}   & \code{nil}      & \code{nil} \\ \hline
				\code{(1 2 3)} & \code{(3 2 0)} & \code{nil}      & \code{nil} \\ \hline
			\end{tabular}
		\end{center}
	\end{table}
\end{task}

\begin{task}
	Напишите необходимые функции, которые обрабатывают таблицу из точечных пар: \code{(страна . столица)}, и возвращают по стране — столицу, а по столице — страну.

	\begin{lstlisting}[language=Lisp, gobble=16]
		(defun get-by-template (getter setter dict value)
		  (reduce
		    #'(lambda (acc cur)
		        (cond (acc acc)
		              ((equal value (funcall getter cur)) (funcall setter cur))
		              (t nil)))
		    dict
		   :initial-value nil))

		(defun get-capital-by-country (dict country)
		  (get-by-template #'car #'cdr dict country))

		(defun get-country-by-capital (dict capital)
		  (get-by-template #'cdr #'car dict capital))
	\end{lstlisting}

	\begin{lstlisting}[language=Lisp, gobble=16]
		(defun get-by-template (getter setter dict value)
		  (cond ((null dict) nil)
		        ((equal (funcall getter dict) value) (funcall setter dict))
		        (t (get-by-template getter setter (cdr dict) value))))

		(defun get-capital-by-country (dict country)
		  (get-by-template #'caar #'cdar dict country))

		(defun get-country-by-capital (dict capital)
		  (get-by-template #'cdar #'caar dict capital))
	\end{lstlisting}

	Пример работы:
	\begin{table}[H]
		\begin{center}
			\begin{tabular}{|r|l|}
				\hline
				\textbf{dict} & \code{((a . b) (b . c) (c . d))} \\ \hline
				\textbf{value} & \code{c} \\ \hline
				\textbf{(get-capital-by-country dict value)} & \code{d} \\ \hline
				\textbf{(get-country-by-capital dict value)} & \code{b} \\ \hline
			\end{tabular}
		\end{center}
	\end{table}
\end{task}

\begin{task}
	Напишите функцию, которая умножает на заданное число-аргумент все числа из заданного списка-аргумента, когда а) все элементы списка — числа, б)  элементы списка — любые объекты.

	\begin{lstlisting}[language=Lisp, gobble=16]
		(defun mult-a (lst number)
		  (mapcar
		    #'(lambda (item)
		        (* item number))
		    lst))

		(defun mult-b (lst number)
		  (mapcar
		    #'(lambda (item)
		        (if (numberp item)
		            (* item number)
		            item))
		    lst))
	\end{lstlisting}

	\begin{lstlisting}[language=Lisp, gobble=16]
		(defun mult-a (lst number)
		  (defun mult-a-process (lst acc)
		    (if lst
		        (mult-a-process
		          (cdr lst)
		          (append acc
		            (list (* (car lst) number))))
		        acc))
		  (mult-a-process lst nil))

		(defun mult-b (lst number)
		  (defun mult-b-process (lst acc)
		    (if lst
		        (mult-b-process
		          (cdr lst)
		          (append acc
		            (if (numberp (car lst))
		                (list (* number (car lst)))
		                (list (car lst)))))
		        acc))
		  (mult-b-process lst nil))
	\end{lstlisting}

	Пример работы:
	\begin{itemize}
		\item \code{(mult-a '(1 2 3 4) 2) -> (2 4 6 8)}
		\item \code{(mult-b '(1 2 a 4) 2) -> (2 4 A 8)}
		\item \code{(mult-a '() 3) -> ()}
		\item \code{(mult-b '() 3) -> ()}
	\end{itemize}
\end{task}

\begin{task}
	Напишите функцию, которая уменьшает на 10 все числа из списка аргумента этой функции.

	\begin{lstlisting}[language=Lisp, gobble=16]
		(defun minus (lst number)
		  (mapcar
		    #'(lambda (item)
		        (- item number))
		    lst))

		(defun minus-ten (lst)
		  (minus lst 10))
	\end{lstlisting}

	\begin{lstlisting}[language=Lisp, gobble=16]
		(defun minus (lst number)
		  (defun minus-process (lst acc)
		    (if lst
		        (minus-process
		          (cdr lst)
		          (append acc
		            (list (- (car lst) number))))
		        acc))
		  (minus-process lst nil))

		(defun minus-10 (lst)
		  (minus lst 10))
	\end{lstlisting}

	Пример работы:
	\begin{itemize}
		\item \code{(minus-10 '(5 10 15)) -> (-5 0 5)}
		\item \code{(minus-10 '()) -> ()}
	\end{itemize}
\end{task}

\begin{task}
	Написать функцию, которая возвращает первый аргумент списка-аргумента, который сам является непустым списком.

	\begin{lstlisting}[language=Lisp, gobble=16]
		(defun get-first-sublist (lst)
		  (reduce
		    #'(lambda (acc cur)
		        (cond (acc acc)
		              ((listp cur) cur)
		              (t nil)))
		    lst
		    :initial-value nil))
	\end{lstlisting}

	\begin{lstlisting}[language=Lisp, gobble=16]
		(defun not-empty-list (lst)
		  (and (listp lst) lst))

		(defun get-first-sublist (lst)
		  (cond ((null lst) nil)
		        ((not-empty-list (car lst)) (car lst))
		        (t (get-first-sublist (cdr lst)))))
	\end{lstlisting}

	Пример работы:
	\begin{itemize}
		\item \code{(get-first-sublist '(1 2 3 (4 5) 6 7 (8 9 (10)))) -> (4 5)}
		\item \code{(get-first-sublist '(1 2 3 4)) -> nil}
	\end{itemize}
\end{task}

\begin{task}
	Написать функцию, которая выбирает из заданного списка только те числа, которые больше 1 и меньше 10.
	(Вариант: между двумя заданными границами)

	\begin{lstlisting}[language=Lisp, gobble=16]
		(defun between (lst a b)
		  (reduce
		    #'(lambda (acc cur)
		        (if (and (< a cur) (< cur b))
		            (append acc (list cur))
		            acc))
		    lst
		    :initial-value nil))

		(defun between-1-10 (lst)
		  (between lst 1 10))
	\end{lstlisting}

	\begin{lstlisting}[language=Lisp, gobble=16]
		(defun between (lst a b)
		  (defun is-between (number)
		    (and (< a number) (< number b)))
		  (defun between-process (lst acc)
		    (if lst
		        (between-process
		          (cdr lst)
		          (if (is-between (car lst))
		              (append acc (list (car lst)))
		              acc))
		        acc))
		  (between-process lst nil))

		(defun between-1-10 (lst)
		  (between lst 1 10))
	\end{lstlisting}

	Пример работы:
	\begin{itemize}
		\item \code{(between-1-10 '(-1 0 1 5 9 10 11 2)) -> (5 9 2)}
		\item \code{(between-1-10 '()) -> ()}
	\end{itemize}
\end{task}

\begin{task}
	Написать функцию, вычисляющую декартово произведение двух своих списков-аргументов.
	(Напомним, что $A\times B$ — это множество всевозможных пар $(a\;b)$, где $a\in A$, $b\in B$)

	\begin{lstlisting}[language=Lisp, gobble=16]
		(defun set-mult (a b)
		  (apply
		    #'append
		        (mapcar
		          #'(lambda (item-a)
		              (mapcar
		                #'(lambda (item-b)
		                    (list item-a item-b))
		                b))
		          a)))
	\end{lstlisting}

	\begin{lstlisting}[language=Lisp, gobble=16]
		(defun set-mult (a b)
		  (defun set-mul-process (x y)
		    (cond ((null x) nil)
		          ((null y) (set-mul-process (cdr x) b))
		          (t (cons (list (car x) (car y))
		                   (set-mul-process x (cdr y))))))
		  (set-mul-process a b))
	\end{lstlisting}

	Пример работы:
	\begin{itemize}
		\item \code{(set-mult '(1 2) '(3 4 5)) -> ((1 3) (1 4) (1 5) (2 3) (2 4) (2 5))}
		\item \code{(set-mult '(1 2 3) '()) -> ()}
	\end{itemize}
\end{task}

\begin{task}
	Почему так реализовано \code{reduce}, в чем причина?

	\begin{lstlisting}[language=Lisp, gobble=16]
		(reduce #'+ 0)  ; 0
		(reduce #'+ ()) ; 0
	\end{lstlisting}

	Из математических соображений
	\begin{equation}
		\sum_{a\in\varnothing} a = 0, \quad \prod_{a\in\varnothing} a = 1.
	\end{equation}
\end{task}

\section*{Теоретическая часть}

\subsection*{Способы организации повторных вычислений в Lisp}

Для организации повторных вычислений в Lisp могут быть использованы функционалы и рекурсия.

\subsection*{Различные способы использования функционалов}

В Lisp используются применяющие и отображающие функционалы, функционалы, являющиеся предикатами, функционалы, использующие предикаты в качестве функционального объекта.

\subsection*{Что такое рекурсия? Способы организации рекурсивных функций}

Рекурсия — это ссылка на определяемый объект во время его определения.

Существуют типы рекурсивных функций: хвостовая, дополняемая, множественная, взаимная рекурсия и рекурсия более высокого порядка.

При организации рекурсии можно использовать как функции с именем, так и локально определенные с помощью лямбда-выражений функции.
Кроме этого, при организации рекурсии можно использовать функционалы или использовать рекурсивную функцию внутри функционала.
При изучении рекурсии рекомендуется организовывать и отлаживать реализацию отдельных подзадач исходной задачи, обращая внимание на эффективность реализации и качество работы, а потом, при необходимости, встраивать эти функции в более крупные, возможно в виде лямбда-выражений.

\subsection*{Способы повышения эффективности реализации рекурсии}

\begin{itemize}
	\item {\bfseries Хвостовая рекурсия}.
	В целях повышения эффективности рекурсивных функций рекомендуется формировать результат не на выходе из рекурсии, а на входе в рекурсию, все действия выполняя до ухода на следующий шаг рекурсии.

	Преобразование не хвостовой рекурсии в хвостовую, возможно путем использования дополнительных параметров.
	В этом случае необходимо использовать функцию-оболочку для запуска рекурсивной функции с начальными значениями дополнительных параметров.

	\item {\bfseries Дополняемая рекурсия}.
	При обращении к рекурсивной функции используется дополнительная функция не в аргументе вызова, а вне его.

	\item {\bfseries Выделяют группу функций множественной рекурсии}.
	На одной ветке происходит сразу несколько рекурсивных вызовов.
	Количество условий выхода также может зависеть от задачи.
\end{itemize}

\end{document}
