\documentclass[a4paper,oneside,12pt]{extreport}

\include{preamble}

\begin{document}

\begin{titlepage}
	{\large % 14pt instead of 12pt
	\onehalfspacing
	\centering

	\begin{wrapfigure}[7]{l}{0.14\linewidth}
		\vspace{3mm}
		\hspace{-10mm}
		\includegraphics[width=0.93\linewidth]{inc/img/bmstu-logo}
	\end{wrapfigure}
	{\singlespacing \footnotesize \bfseries Министерство науки и высшего образования Российской Федерации\\Федеральное государственное бюджетное образовательное учреждение\\высшего образования\\<<Московский государственный технический университет\\имени Н.~Э.~Баумана\\ (национальный исследовательский университет)>>\\(МГТУ им. Н.~Э.~Баумана)\\}

	\vspace{-2.2mm}
	\vhrulefill{0.9mm}\\
	\vspace{-7.5mm}
	\vhrulefill{0.2mm}\\
	\vspace{2mm}

	{\doublespacing \small \raggedright ФАКУЛЬТЕТ \hspace{25mm} «Информатика и системы управления»\\
	КАФЕДРА \hspace{5mm} «Программное обеспечение ЭВМ и информационные технологии»\\}

	\vspace{30mm}

	\textbf{ОТЧЁТ}\\
	По лабораторной работе №3\\
	По курсу: «Функциональное и логическое программирование»\\
	%Тема: «Тема работы»\\

	\vspace{60mm}

	\hspace{70mm} Студент:       \hfill Керимов~А.~Ш.\\
	\hspace{70mm} Группа:        \hfill ИУ7-64Б\\
	\hspace{70mm} Преподаватели: \hfill Толпинская~Н.~Б.,\\
	                             \hfill Строганов~Ю.~В.\\

	\vfill

	Москва\\
	\the\year\\}
\end{titlepage}


\begin{task}
	Пусть \code{(setf lst1 '(a b)) (setf lst2 '(c d))}.
	Каковы результаты вычисления следующих выражений?

	\begin{enumerate}
		\item \begin{lstlisting}[style=lispinline, gobble=24]
			(cons lst1 lst2) ; ((A B) C D)
		\end{lstlisting}

		\item \begin{lstlisting}[style=lispinline, gobble=24]
			(list lst1 lst2) ; ((A B) (C D))
		\end{lstlisting}

		\item \begin{lstlisting}[style=lispinline, gobble=24]
			(append lst1 lst2) ; (A B C D)
		\end{lstlisting}
	\end{enumerate}
\end{task}

\begin{task}
	Каковы результаты вычисления следующих выражений?

	\begin{AutoMultiColEnumerate}
		\item \begin{lstlisting}[style=lispinline, gobble=24]
			(reverse ()) ; NIL
		\end{lstlisting}

		\item \begin{lstlisting}[style=lispinline, gobble=24]
			(last ()) ; NIL
		\end{lstlisting}

		\item \begin{lstlisting}[style=lispinline, gobble=24]
			(reverse '(a)) ; (A)
		\end{lstlisting}

		\item \begin{lstlisting}[style=lispinline, gobble=24]
			(last '(a)) ; (A)
		\end{lstlisting}

		\item \begin{lstlisting}[style=lispinline, gobble=24]
			(reverse '((a b c))) ; ((A B C))
		\end{lstlisting}

		\item \begin{lstlisting}[style=lispinline, gobble=24]
			(last '((a b c))) ; ((A B C))
		\end{lstlisting}
	\end{AutoMultiColEnumerate}
\end{task}

\begin{task}
	Написать, по крайней мере, два варианта функции, которая возвращает последний элемент своего списка-аргумента.

	\begin{lstlisting}[language=Lisp, gobble=16]
		(defun my-last1 (x)
		  (car (reverse x)))

		(defun my-last2 (x)
		  (cond ((null (cdr x)) (car x))
		        (T (my_last2 (cdr x)))))

		(defun my-last3 (x)
		  (reduce (lambda (p n) n)))
	\end{lstlisting}
\end{task}

\begin{task}
	Написать, по крайней мере, два варианта функции, которая возвращает свой список-аргумент без последнего элемента.

	\begin{lstlisting}[language=Lisp, gobble=16]
		(defun lastout1 (x)
		  (reverse (cdr (reverse x))))

		(defun lastout2 (x)
		  (butlast x))
	\end{lstlisting}
\end{task}

\begin{task}
	Написать простой вариант игры в кости, в котором бросаются две правильные кости.
	Если сумма выпавших очков равна 7 или 11 — выигрыш, если выпало (1, 1) или (6, 6) — игрок получает право снова бросить кости, во всех остальных случаях ход переходит ко второму игроку, но запоминается сумма выпавших очков.
	Если второй игрок не выигрывает абсолютно, то выигрывает тот игрок, у которого больше очков.

	\begin{lstlisting}[language=Lisp, gobble=16]
		(defun throw-one () (+ (random 6) 1))

		(defun throw-two () (list (throw-one) (throw-one)))

		(defun next-try (dices)
		  (or (equal dices '(1 1))
		      (equal dices '(6 6))))

		(defun sum-dices (dices)
		  (+ (car dices) (cadr dices)))

		(defun is-win (dices)
		  (let ((rate (sum-dices dices)))
		       (or (= rate 7) (= rate 11))))

		(defun play-round ()
		  (let ((dices (throw-two)))
		       (cond
		         ((is-win dices) (list 0 dices))
		         ((next-try dices) (play-round))
		         (T (list (sum-dices dices) dices)))))

		(defun print-sum (d1 d2 msg)
		  (format nil "Player 1 - ~{~a~^ ~}~%Player 2 - ~{~a~^ ~}~%~a"
		    (cdr d1) (cdr d2) msg))

		(defun play ()
		  (let ((d1 (play-round))
		        (d2 (play-round)))
		       (cond
		         ((= (car d1) 0) (print-sum d1 d2 "Player 1 won!"))
		         ((= (car d1) 0) (print-sum d1 d2 "Player 2 won!"))
		         ((> (car d1) (car d2)) (print-sum d1 d2 "Player 1 won (sum)!"))
		         ((< (car d1) (car d2)) (print-sum d1 d2 "Player 2 won (sum)!"))
		         (T (print-sum d1 d2 "Draw!")))))
	\end{lstlisting}
\end{task}

\end{document}
