\documentclass[a4paper,oneside,12pt]{extreport}

\include{preamble}

\begin{document}

\begin{titlepage}
	{\large % 14pt instead of 12pt
	\onehalfspacing
	\centering

	\begin{wrapfigure}[7]{l}{0.14\linewidth}
		\vspace{3mm}
		\hspace{-10mm}
		\includegraphics[width=0.93\linewidth]{inc/img/bmstu-logo}
	\end{wrapfigure}
	{\singlespacing \footnotesize \bfseries Министерство науки и высшего образования Российской Федерации\\Федеральное государственное бюджетное образовательное учреждение\\высшего образования\\<<Московский государственный технический университет\\имени Н.~Э.~Баумана\\ (национальный исследовательский университет)>>\\(МГТУ им. Н.~Э.~Баумана)\\}

	\vspace{-2.2mm}
	\vhrulefill{0.9mm}\\
	\vspace{-7.5mm}
	\vhrulefill{0.2mm}\\
	\vspace{2mm}

	{\doublespacing \small \raggedright ФАКУЛЬТЕТ \hspace{25mm} «Информатика и системы управления»\\
	КАФЕДРА \hspace{5mm} «Программное обеспечение ЭВМ и информационные технологии»\\}

	\vspace{30mm}

	\textbf{ОТЧЁТ}\\
	По лабораторной работе №3\\
	По курсу: «Функциональное и логическое программирование»\\
	%Тема: «Тема работы»\\

	\vspace{60mm}

	\hspace{70mm} Студент:       \hfill Керимов~А.~Ш.\\
	\hspace{70mm} Группа:        \hfill ИУ7-64Б\\
	\hspace{70mm} Преподаватели: \hfill Толпинская~Н.~Б.,\\
	                             \hfill Строганов~Ю.~В.\\

	\vfill

	Москва\\
	\the\year\\}
\end{titlepage}


\section*{Практическая часть}

\begin{task}
	Дана функция \code{(defun mystery (x) (list (second x) (first x)))}.
	Какие результаты вычисления следующих выражений?
	\begin{AutoMultiColEnumerate}
		\item \begin{lstlisting}[style=lispinline, gobble=24]
			(mystery (one two))
			; (two one)
		\end{lstlisting}

		\item \begin{lstlisting}[style=lispinline, gobble=24]
			(mystery one 'two))
			; Invalid number of arguments: 2.
		\end{lstlisting}

		\item \begin{lstlisting}[style=lispinline, gobble=24]
			(mystery 'free)
			; The value FREE is not LIST.
		\end{lstlisting}

		\item \begin{lstlisting}[style=lispinline, gobble=24]
			(mystery (last 'one 'two))
			; The value ONE is not LIST.
		\end{lstlisting}
	\end{AutoMultiColEnumerate}
\end{task}

\begin{task}
	Написать функцию, которая переводит температуру в системе Фаренгейта температуру по Цельсию \code{(defun f-to-c (temp) ...)}.

	\begin{lstlisting}[language=Lisp, gobble=16]
		(defun f-to-c (temp)
			(*
				(/ 5 9)
				(- temp 32.0)
			)
		)
	\end{lstlisting}
\end{task}

\begin{task}
	Что получится при вычисления каждого из выражений?
	\begin{AutoMultiColEnumerate}
		\item \begin{lstlisting}[style=lispinline, gobble=24]
			(list 'cons T Nil))
			; (cons T Nil)
		\end{lstlisting}

		\item \begin{lstlisting}[style=lispinline, gobble=24]
			(cons 'a (b c))
			; Undefined function: b.
		\end{lstlisting}

		\item \begin{lstlisting}[style=lispinline, gobble=24]
			(eval (eval (list 'cons T Nil)))
			; Undefined functionz: T
		\end{lstlisting}

		\item \begin{lstlisting}[style=lispinline, gobble=24]
			(apply #'cons '(T Nil))	
			; (T)
		\end{lstlisting}

		\item \begin{lstlisting}[style=lispinline, gobble=24]
			(list 'eval Nil)
			; (eval Nil)
		\end{lstlisting}

		\item \begin{lstlisting}[style=lispinline, gobble=24]
			(eval (list 'cons T Nil))
			; (T)
		\end{lstlisting}

		\item \begin{lstlisting}[style=lispinline, gobble=24]
			(eval Nil)
			; Nil
		\end{lstlisting}

		\item \begin{lstlisting}[style=lispinline, gobble=24]
			(eval (list 'eval NIL))
			; Nil
		\end{lstlisting}
	\end{AutoMultiColEnumerate}
\end{task}

\begin{task}
	Написать функцию, вычисляющую катет по заданной гипотенузе и другому катету прямоугольного треугольника, и составить диаграмму её вычисления.

	\begin{lstlisting}[language=Lisp, gobble=16]
		(defun cathetus (c b)
			(sqrt
				(-
					(* c c)
					(* b b)
				)
			)
		)
	\end{lstlisting}
\end{task}

\begin{task}
	Написать функцию, вычисляющую площадь трапеции по ее основаниям и высоте, и составить диаграмму ее вычисления.
	\begin{lstlisting}[language=Lisp, gobble=16]
		(defun trapezoid-area (base1 base2 height)
			(/
				(*
					(+ base1 base2)
					height
				)
				2.0
			)
		)
	\end{lstlisting}
\end{task}

\section*{Теоретическая часть}

\subsection*{Как синтаксически представляется программа на Lisp}

\subsection*{Как трактуются элементы списка?}

\subsection*{Порядок реализации программы}

\end{document}
