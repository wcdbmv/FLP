\documentclass[a4paper,oneside,12pt]{extreport}

\include{preamble}

\begin{document}

\begin{titlepage}
	{\large % 14pt instead of 12pt
	\onehalfspacing
	\centering

	\begin{wrapfigure}[7]{l}{0.14\linewidth}
		\vspace{3mm}
		\hspace{-10mm}
		\includegraphics[width=0.93\linewidth]{inc/img/bmstu-logo}
	\end{wrapfigure}
	{\singlespacing \footnotesize \bfseries Министерство науки и высшего образования Российской Федерации\\Федеральное государственное бюджетное образовательное учреждение\\высшего образования\\<<Московский государственный технический университет\\имени Н.~Э.~Баумана\\ (национальный исследовательский университет)>>\\(МГТУ им. Н.~Э.~Баумана)\\}

	\vspace{-2.2mm}
	\vhrulefill{0.9mm}\\
	\vspace{-7.5mm}
	\vhrulefill{0.2mm}\\
	\vspace{2mm}

	{\doublespacing \small \raggedright ФАКУЛЬТЕТ \hspace{25mm} «Информатика и системы управления»\\
	КАФЕДРА \hspace{5mm} «Программное обеспечение ЭВМ и информационные технологии»\\}

	\vspace{30mm}

	\textbf{ОТЧЁТ}\\
	По лабораторной работе №3\\
	По курсу: «Функциональное и логическое программирование»\\
	%Тема: «Тема работы»\\

	\vspace{60mm}

	\hspace{70mm} Студент:       \hfill Керимов~А.~Ш.\\
	\hspace{70mm} Группа:        \hfill ИУ7-64Б\\
	\hspace{70mm} Преподаватели: \hfill Толпинская~Н.~Б.,\\
	                             \hfill Строганов~Ю.~В.\\

	\vfill

	Москва\\
	\the\year\\}
\end{titlepage}


\section*{Практическая часть}

\begin{task}
	Составить диаграмму вычисления следующих выражений:
	\begin{AutoMultiColEnumerate}
		\item \begin{lstlisting}[style=lispinline, gobble=24]
			(equal 3 (abs -3))       ; T
		\end{lstlisting}

		\item \begin{lstlisting}[style=lispinline, gobble=24]
			(equal (* 2 3) (+ 7 2))  ; Nil
		\end{lstlisting}

		\item \begin{lstlisting}[style=lispinline, gobble=24]
			(equal (+ 1 2) 3)        ; T
		\end{lstlisting}

		\item \begin{lstlisting}[style=lispinline, gobble=24]
			(equal (- 7 3) (* 3 2))  ; Nil
		\end{lstlisting}

		\item \begin{lstlisting}[style=lispinline, gobble=24]
			(equal (* 4 7) 21)       ; Nil
		\end{lstlisting}

		\item \begin{lstlisting}[style=lispinline, gobble=24]
			(equal (abs (- 2 4)) 3)) ; Nil
		\end{lstlisting}
	\end{AutoMultiColEnumerate}
\end{task}

\begin{task}
	Написать функцию, вычисляющую гипотенузу прямоугольного треугольника по заданным катетам и составить диаграмму её вычисления.

	\begin{lstlisting}[language=Lisp, gobble=16]
		(defun hypotenuse (a b)
			(sqrt (+ (* a a) (* b b)))
		)
	\end{lstlisting}
\end{task}


\begin{task}
	Написать функцию, вычисляющую объём параллелепипеда по 3-м его сторонам, и составить диаграмму её вычисления.

	\begin{lstlisting}[language=Lisp, gobble=16]
		(defun parallelepiped-volume (a b c)
			(* a b c)
		)
	\end{lstlisting}
\end{task}

\begin{task}
	Каковы результаты вычисления следующих выражений?
	\begin{AutoMultiColEnumerate}
		\item \begin{lstlisting}[style=lispinline, gobble=24]
			(list 'a 'b c)
			; The variable c is unbound.
		\end{lstlisting}

		\item \begin{lstlisting}[style=lispinline, gobble=24]
			(cons 'a (b c))
			; Undefined function: b.
		\end{lstlisting}

		\item \begin{lstlisting}[style=lispinline, gobble=24]
			(cons 'a '(b c))
			; (a b c)
		\end{lstlisting}

		\item \begin{lstlisting}[style=lispinline, gobble=24]
			(caddr (1 2 3 4 5))
			; Illegal function call.
		\end{lstlisting}

		\item \begin{lstlisting}[style=lispinline, gobble=24]
			(cons 'a 'b 'c)
			; Invalid number of arguments: 3.
		\end{lstlisting}

		\item \begin{lstlisting}[style=lispinline, gobble=24]
			(list 'a (b c))
			; Undefined function: b.
		\end{lstlisting}

		\item \begin{lstlisting}[style=lispinline, gobble=24]
			(list a '(b c))
			; The variable a is unbound.
		\end{lstlisting}

		\item \begin{lstlisting}[style=lispinline, gobble=24]
			(list (+ 1 '(length '(1 2 3))))
			; The value (length '(1 2 3)) is not of type number.
		\end{lstlisting}
	\end{AutoMultiColEnumerate}
\end{task}

\begin{task}
	Написать функцию \code{longer-than} от двух списков-аргументов, которая возвращает \code{T}, если первый аргумент имеет большую длину.

	\begin{lstlisting}[language=Lisp, gobble=16]
		(defun longer-than (a b)
			(> (list-length a) (list-length b))
		)
	\end{lstlisting}
\end{task}

\begin{task}
	Каковы результаты вычисления следующих выражений?
	\begin{AutoMultiColEnumerate}
		\item \begin{lstlisting}[style=lispinline, gobble=24]
			(cons 3 (list 5 6))
			; (3 5 6)
		\end{lstlisting}

		\item \begin{lstlisting}[style=lispinline, gobble=24]
			(cons 3 '(list 5 6))
			; (3 list 5 6)
		\end{lstlisting}

		\item \begin{lstlisting}[style=lispinline, gobble=24]
			(list 3 'from 9 'gives (- 9 3))
			; (3 from 9 gi 6)
		\end{lstlisting}

		\item \begin{lstlisting}[style=lispinline, gobble=24]
			(+ (length '(1 foo 2 too)) (car '(21 22 23)))
			; 25
		\end{lstlisting}

		\item \begin{lstlisting}[style=lispinline, gobble=24]
			(cdr '(cons is short for ans))
			; (is short for ans)
		\end{lstlisting}

		\item \begin{lstlisting}[style=lispinline, gobble=24]
			(car (list one two))
			; The variable one is unbound.
		\end{lstlisting}

		\item \begin{lstlisting}[style=lispinline, gobble=24]
			(car (list 'one 'two))
			; one
		\end{lstlisting}
	\end{AutoMultiColEnumerate}
\end{task}

\section*{Теоретическая часть}

\subsection*{Базис Lisp}

Базис Lisp предельно лаконичен — атомы и структуры из простейших бинарных узлов плюс несколько базовых функций и функционалов.
Базис содержит встроенные (примитивные) функции, которые анализируют, строят и разбирают любые структурные значения (\code{atom}, \code{eq}, \code{cons}, \code{car}, \code{cdr}), и встроенные специальные функции и функционалы, которые управляют обработкой структур, представляющих вычисляемые выражения (\code{quote}, \code{cond}, \code{lambda}, \code{eval}).

\subsection*{Варианты классификаций функций Lisp}

Классификация функций:
\begin{enumerate}
	\item чистые математические функции (имеют фиксированное количество аргументов и один результат);
	\item формы (имеют произвольное количество аргументов или эти аргументы обрабатываются не все одинаково);
	\item функциональные (в качестве одного из аргументов принимают описание функции).
\end{enumerate}

Классификация базисных функций:
\begin{enumerate}
	\item функции-селекторы: \code{car}, \code{cdr};
	\item функции-конструкторы: \code{cons}, \code{list};
	\item функции-предикаты: \code{atom}, \code{null}, \code{listp}, \code{consp};
	\item функции сравнения: \code{eq}, \code{eql}, \code{equal}, \code{equalp}.
\end{enumerate}

\subsection*{Как представляются списки в ОП}

Любая непустая структура Lisp в памяти представляется списковой ячейкой, хранящей два указателя: на голову (первый элемент) и хвост — всё остальное.

\subsection*{Как работают car и cdr}

Функция \code{car} обеспечивает доступ к первому элементу списка — его «голове», а функция \code{cdr} — к укороченному на один элемент списку - его «хвосту», т.~е. к тому, что остается после удаления головы.

\subsection*{Отличие работы list и cons}

Функция \code{cons} строит списки из бинарных узлов, заполняя их парами объектов, являющихся значениями пары её аргументов.
Первый аргумент произвольного вида размещается в левой части бинарного узла, а второй, являющийся списком, — в правой.

\begin{lstlisting}[style=lispinline, gobble=8]
	(cons 'a 'b)   ; (a . b)
	(cons 'a '(b)) ; (a b)
\end{lstlisting}

Функция \code{list} строит список, не является чистой, так как имеет произвольное количество аргументов.
\begin{lstlisting}[style=lispinline, gobble=8]
	(list 'a 'b 'c)   ; (a b c)
	(list 'a '(b c)) ; (a (b c))
\end{lstlisting}

\end{document}
