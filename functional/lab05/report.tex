\documentclass[a4paper,oneside,12pt]{extreport}

\include{preamble}

\begin{document}

\begin{titlepage}
	{\large % 14pt instead of 12pt
	\onehalfspacing
	\centering

	\begin{wrapfigure}[7]{l}{0.14\linewidth}
		\vspace{3mm}
		\hspace{-10mm}
		\includegraphics[width=0.93\linewidth]{inc/img/bmstu-logo}
	\end{wrapfigure}
	{\singlespacing \footnotesize \bfseries Министерство науки и высшего образования Российской Федерации\\Федеральное государственное бюджетное образовательное учреждение\\высшего образования\\<<Московский государственный технический университет\\имени Н.~Э.~Баумана\\ (национальный исследовательский университет)>>\\(МГТУ им. Н.~Э.~Баумана)\\}

	\vspace{-2.2mm}
	\vhrulefill{0.9mm}\\
	\vspace{-7.5mm}
	\vhrulefill{0.2mm}\\
	\vspace{2mm}

	{\doublespacing \small \raggedright ФАКУЛЬТЕТ \hspace{25mm} «Информатика и системы управления»\\
	КАФЕДРА \hspace{5mm} «Программное обеспечение ЭВМ и информационные технологии»\\}

	\vspace{30mm}

	\textbf{ОТЧЁТ}\\
	По лабораторной работе №3\\
	По курсу: «Функциональное и логическое программирование»\\
	%Тема: «Тема работы»\\

	\vspace{60mm}

	\hspace{70mm} Студент:       \hfill Керимов~А.~Ш.\\
	\hspace{70mm} Группа:        \hfill ИУ7-64Б\\
	\hspace{70mm} Преподаватели: \hfill Толпинская~Н.~Б.,\\
	                             \hfill Строганов~Ю.~В.\\

	\vfill

	Москва\\
	\the\year\\}
\end{titlepage}


\section*{Практическая часть}

\begin{task}
	Написать функцию, которая принимает целое число и возвращает первое чётное число, не меньшее аргумента.

	\begin{lstlisting}[language=Lisp, gobble=16]
		(defun f (x)
		  (+ x (mod x 2)))
	\end{lstlisting}
\end{task}

\begin{task}
	Написать функцию, которая принимает число и возвращает число того же знака, но с модулем на 1 больше модуля аргумента.

	\begin{lstlisting}[language=Lisp, gobble=16]
		(defun f (x)
		  (+ x (signum x)))
	\end{lstlisting}
\end{task}

\begin{task}
	Написать функцию, которая принимает два числа и возвращает список из этих чисел, расположенных по возрастанию.

	\begin{lstlisting}[language=Lisp, gobble=16]
		(defun f (a b)
		  (if (< a b)
		      (list a b)
		      (list b a))
	\end{lstlisting}
\end{task}

\begin{task}
	Написать функцию, которая принимает три числа и возвращает T только тогда, когда первое число расположено между вторым и третьим.

	\begin{lstlisting}[language=Lisp, gobble=16]
		(defun f (a b c)
		  (and (<= a b) (<= b c))
	\end{lstlisting}
\end{task}

\begin{task}
	Каков результат вычисления следующих выражений?
	\begin{AutoMultiColEnumerate}
		\item \begin{lstlisting}[style=lispinline, gobble=24]
			(and 'fee 'fie 'foe)
			; foe
		\end{lstlisting}

		\item \begin{lstlisting}[style=lispinline, gobble=24]
			(or 'fee 'fie 'foe)
			; fee
		\end{lstlisting}

		\item \begin{lstlisting}[style=lispinline, gobble=24]
			(and (equal 'abc 'abc) 'yes)
			; yes
		\end{lstlisting}

		\item \begin{lstlisting}[style=lispinline, gobble=24]
			(or nil 'fie 'foe)
			; fie
		\end{lstlisting}

		\item \begin{lstlisting}[style=lispinline, gobble=24]
			(and nil 'fie 'foe)
			; nil
		\end{lstlisting}

		\item \begin{lstlisting}[style=lispinline, gobble=24]
			(or (equal 'abc 'abc) 'yes)
			; T
		\end{lstlisting}
	\end{AutoMultiColEnumerate}
\end{task}

\begin{task}
	Написать предикат, который принимает два числа-аргумента и возвращает T, если первое число не меньше второго.

	\begin{lstlisting}[language=Lisp, gobble=16]
		(defun f (a b)
		  (>= a b))
	\end{lstlisting}
\end{task}

\begin{task}
	Какой из следующих двух вариантов предикатов ошибочен и почему?
	\begin{AutoMultiColEnumerate}
		\item \begin{lstlisting}[style=lispinline, gobble=24]
			(defun pred1 (x)
			  (and (numberp x) (plusp x)))
		\end{lstlisting}

		\item \begin{lstlisting}[style=lispinline, gobble=24]
			(defun pred2 (x)
			  (and (plusp x) (numberp x)))
		\end{lstlisting}
	\end{AutoMultiColEnumerate}

	Ошибочен предикат \code{pred2}, так как сначала вычисляется \code{plusp}, который может вернуть ошибку в случае, если \code{x} не будет действительным числом.
\end{task}

\begin{task}
	Решить задачу 4, используя для её решения конструкции \code{if}, \code{cond}, \code{and}/\code{or}.

	\begin{lstlisting}[language=Lisp, gobble=16]
		(defun f (a b c)
		  (if (<= a b)
		      (if (<= b c)
		          t
		          nil)
		      nil))

		(defun f (a b c)
		  (cond ((<= a b)
		         (cond ((<= b c) t)
		                (t nil)))
		         (t nil)))
	\end{lstlisting}
\end{task}

%\section*{Теоретическая часть}

%\subsection*{Определение атома, представление его в памяти}

%\subsection*{Самовычислимый атом}

%\subsection*{Локальное и глобальное определение значений атома}

%\subsection*{eval и quote}

\end{document}
